% \input{warning}

\PassOptionsToPackage{hyphens}{url}
\usepackage[hyphens]{url}
\PassOptionsToPackage{breaklinks,colorlinks}{hyperref}
\usepackage[breaklinks,colorlinks]{hyperref}
\PassOptionsToPackage{usenames,dvipsnames}{xcolor}
\usepackage[usenames,dvipsnames]{xcolor}
\hypersetup{
  colorlinks,
  linkcolor={red!50!black},
  citecolor={blue!50!black},
  urlcolor={blue!50!black}
}
\usepackage{amsmath,amsopn,amssymb}
\usepackage{colortbl}
\usepackage{endnotes,microtype,xspace,graphicx,fancyvrb,multirow}
\usepackage{booktabs}
\usepackage{array,underscore,relsize}
\usepackage[T1]{fontenc}
\usepackage{times}
\usepackage{fancyhdr,lastpage}
\usepackage[inline]{enumitem}
\usepackage[labelfont=bf,font=small,skip=5pt]{caption}
% NOTE: either subfigure of subcaption; not both
%\usepackage{subfigure}
\usepackage[belowskip=0pt,aboveskip=2pt]{subcaption}
\pagestyle{fancy}
\fancyhf{}
\renewcommand{\headrulewidth}{0pt}
\cfoot{\thepage}
\usepackage[normalem]{ulem}

% for math macro and numbers
\usepackage{fp}
\usepackage{siunitx}

% pseudo code
%%%% minted
\makeatletter
\@namedef{ver@lineno.sty}{9999/12/31}
\@namedef{opt@lineno.sty}{}
\makeatother
\usepackage{minted}
%%%% other method

% \usepackage{algorithm}
%\usepackage{algorithm2e}
% \usepackage[noend]{algpseudocode}
% \algrenewcommand\algorithmicindent{0.75em}

%%%%

% balance bibliography
\usepackage{balance}
\usepackage{multicol}


% use \num{123456} -> 123,456
\sisetup{group-separator={,},group-minimum-digits={3},output-decimal-marker={.}}

\usepackage[super]{nth}
\usepackage{pifont}
\usepackage{import}
\usepackage{placeins}
\usepackage{longtable}
\usepackage{supertabular}
\usepackage{xr}
\usepackage{pgffor}